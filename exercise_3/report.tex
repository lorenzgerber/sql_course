\documentclass[a4paper,11pt,twoside]{article}
%\documentclass[a4paper,11pt,twoside,se]{article}

\usepackage{UmUStudentReport}
\usepackage{verbatim}   % Multi-line comments using \begin{comment}
\usepackage{courier}    % Nicer fonts are used. (not necessary)
\usepackage{pslatex}    % Also nicer fonts. (not necessary)
\usepackage[pdftex]{graphicx}   % allows including pdf figures
\usepackage{listings}
\usepackage{pgf-umlcd}
\usepackage{blindtext}
\usepackage{enumitem}
\usepackage{amsfonts}
\usepackage{amssymb}
%\usepackage{mathtools}

%\usepackage{lmodern}   % Optional fonts. (not necessary)
%\usepackage{tabularx}
%\usepackage{microtype} % Provides some typographic improvements over default settings
%\usepackage{placeins}  % For aligning images with \FloatBarrier
%\usepackage{booktabs}  % For nice-looking tables
%\usepackage{titlesec}  % More granular control of sections.

% DOCUMENT INFO
% =============
\department{Department of Computing Science}
\coursename{Parallel Programming 7.5 p}
\coursecode{5DV152}
\title{Exercises, Chapter/Topic 1}
\author{Lorenz Gerber ({\tt{dv15lgr@cs.umu.se}} {\tt{lozger03@student.umu.se}})}
\date{2017-02-21}
%\revisiondate{2016-01-18}
\instructor{Jan Erik Moström / Michael Minnert}


% DOCUMENT SETTINGS
% =================
\bibliographystyle{plain}
%\bibliographystyle{ieee}
\pagestyle{fancy}
\raggedbottom
\setcounter{secnumdepth}{2}
\setcounter{tocdepth}{2}
%\graphicspath{{images/}}   %Path for images

\usepackage{float}
\floatstyle{ruled}
\newfloat{listing}{thp}{lop}
\floatname{listing}{Listing}



% DEFINES
% =======
%\newcommand{\mycommand}{<latex code>}
%\DeclarePairedDelimiter{\ceil}{\lceil}{\rceil}

% DOCUMENT
% ========
\begin{document}
\lstset{language=C}
\maketitle
\thispagestyle{empty}
\newpage
\tableofcontents
\thispagestyle{empty}
\newpage

\clearpage
\pagenumbering{arabic}

\section{Query 1}
Find the code of each airport which is located either in Greece or else in Germany
\subsubsection*{Relational Algebra}
$X_1 \leftarrow \sigma_{(Country = `Germany') \land (Country = `France')}(Airport)$\\
$X_2 \leftarrow \pi_{(Code)}(X_1)$\\

\subsubsection*{Relational Tuple Calculus}
$\{a.Code|Airport(a)$\\
$\land((a.Country = `Germany`) \lor (a.Country = `France`))\}$

\section{Query 2}
Find the name and abbreviation of each airline which has a flight with destination the airport with the code `TXL' but no flight with destination the airport with code `SXF'
\subsubsection*{Relational Algebra}
$X_1 \leftarrow \pi_{Airline}(\sigma_{Destination=`TXL'}(Flight))$\\
$X_2 \leftarrow \pi_{Airline}(\sigma_{Destination=`SXF'}(Flight))$\\
$X_5 \leftarrow X_1 \backslash X_2$\\
$X_6 \leftarrow X_5 \Join_{Airline=Abbreviation} Airline$\\
$X_7 \leftarrow \pi_{Name, Abbreviation}(X_6)$\\

\subsubsection*{Relational Tuple Calculus}
$\{a.Name, a.Abbreviation|Airline(a)$\\
$\land \}$


\section{Query 3}
Find the name and abbreviation of those airlines which do not have any flights to an airport in Germany or France
 
\section{Relational Algebra}

\section{Relational Tuple Calculus}

\section{Query 4}
Find the codes of those airports which have flights to every airport in France. (Note that no French airport will normally qualify because, for example, there is no flight from `CDG' to `CDG'.)
\section{Relational Algebra}

\section{Relational Tuple Calculus}

\section{Query 5}
Find the codes of those airports which have departures (i.e. fligths with origin at that airport) for exactly two distinct airines.
\section{Relational Algebra}

\section{Relational Tuple Calculus}


\addcontentsline{toc}{section}{\refname}
\bibliography{references}

\end{document}
